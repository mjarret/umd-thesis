\appendix

\chapter{A sufficient condition for some node of $\mathbf{u}(\mu)$ to separate adjacent nodes of $\mathbf{u}(\lambda)$}\label{app:Gantmakher}

In this section we give proof of \Cref{thm:Gantmakher}. The following proof is adapted directly from Gantmakher and Kre\u{i}n\cite{gantmakher2002oscillation}, however we consider vectors $\mathbf{u}(\lambda)$ and $\mathbf{u}(\mu)$ which need not be eigenvectors of the same matrix. Rather, we require only that these vectors satisfy \cref{eqn:recurrence}.

    \Gantmakher*

    \begin{proof}
      First, we consider the extension of vectors $\mathbf{u}(\mu;\alpha)$,$\mathbf{u}(\lambda;\beta)$. From \cref{eqn:recurrence}  we take $W_1 \mapsto W_1+1$ and, for $\mathbf{u}(\lambda;\beta)$ get
      \begin{equation}\label{eqn:recurrence_node}
      	\left(1+W_1 - \lambda\right)u_1(\lambda) = u_{2}(\lambda) + u_{0}(\lambda)
      \end{equation}
      and similarly for $\mathbf{u}(\mu;\alpha)$. Here, to maintain consistency between \cref{eqn:recurrence,eqn:recurrence_node} we require that in \cref{eqn:recurrence_node} $u_0=0$. Thus, $u_0$ is a node of $\mathbf{u}$. We similarly treat $u_{N+1}$ as a node. Further, since we have shifted $W_1,W_{N}$ by constants, \cref{eqn:theta} is unaltered. Hence, \cref{eqn:Casoratian} is unchanged and we can proceed with the proof.

      Let $\eta \in [m-1,m)$ and $\xi \in (n,n+1]$ be successive nodes of $\mathbf{u}(\lambda;\beta)$ with $\eta < \xi$. Without loss of generality, we assume that $u_i(\lambda;\beta) > 0 \; \forall \; i \in \intset{m}{n}$. Then,

      \begin{equation}\label{eqn:Appendix_conditional}
	\begin{cases}
	      (m-\eta)u_{m-1}(\lambda;\beta) + (\eta - m + 1)u_{m}(\lambda;\beta) &= 0 \\
	      (n+1-\xi)u_n(\lambda;\beta) + (\xi-n)u_{n+1}(\lambda;\beta) &= 0
	\end{cases}
      \end{equation}

      Now, again without loss of generality, we assume that $u_i(\mu;\alpha) > 0 \; \forall \; i \in \intset{m}{n}$. Hence, if $\mathbf{u}(\mu;\alpha)$ also has no nodes in $(m,n)$, we get that

      \begin{equation}\label{eqn:Appendix_conditional2}
	\begin{cases}
	      (m-\eta)u_{m-1}(\mu;\alpha) + (\eta-m+1)u_{m}(\mu;\alpha) &\geq 0 \\
	      (n+1-\xi)u_n(\mu;\alpha) + (\xi-n)u_{n+1}(\mu;\alpha) &\geq 0
	\end{cases}
      \end{equation}

      Combining \cref{eqn:Appendix_conditional,eqn:Appendix_conditional2} yields the inequalities
      \begin{align}
	      \label{eqn:Appendix_inequalities1} w_{m-1}\big(\mathbf{u}(\mu;\beta),\mathbf{u}(\lambda;\alpha)\big) &\leq 0 \\
	      \label{eqn:Appendix_inequalities2} w_{n}\big(\mathbf{u}(\mu;\beta),\mathbf{u}(\lambda;\alpha)\big) &\geq 0
      \end{align}

      Recall from \cref{eqn:Casoratian} that
      \begin{equation}
	      \Delta w_{i-1}\big(\mathbf{u}(\mu;\beta),\mathbf{u}(\lambda;\alpha)\big) = \Theta_{W,i}(\mu-\lambda;\beta,\alpha) u_{i}(\mu;\beta)u_i(\lambda;\alpha)
      \end{equation}
      where by summing both sides,
      \begin{equation}\label{eq:Appendix_Casoratian}
	      w_{n}\big(\mathbf{u}(\mu;\beta),\mathbf{u}(\lambda;\alpha)\big)-w_{m-1}\big(\mathbf{u}(\mu;\beta),\mathbf{u}(\lambda;\alpha)\big) = \sum_{i=m}^{n}\Theta_{W,i}(\mu-\lambda;\beta,\alpha) u_{i}(\mu;\beta)u_i(\lambda;\alpha)
      \end{equation}
      Thus, by \cref{eqn:Appendix_inequalities1,eqn:Appendix_inequalities2} we have that the left-hand side of \cref{eq:Appendix_Casoratian} is non-negative. Then, by our choice of $u_i(\lambda;\alpha),u_i(\mu;\beta)>0 \; \forall \; i \in \intset{m}{n}$, we see that if $\Theta_{W,i}(\mu-\lambda;\beta,\alpha) \leq 0 \; \forall \; i \in \intset{m}{n}$ with at least some $i\in \intset{m}{n}$ such that $\Theta_{W,i}(\mu-\lambda;\beta,\alpha) < 0$ we arrive at a contradiction.
    \end{proof}

\chapter{For $\mathbf{H}_{\alpha U}(\mathbb{P}_N)$, the node of $\mathbf{u}(\lambda_2)$ shifts left with increasing $\alpha$.}\label{app:Sturm-Picone}

\begin{thm}
	Let $\mathbf{H}_{\alpha U}(\mathbb{P}_N)$ be defined as in \Cref{lem:node_left}. Then, the node of $\mathbf{u}(\lambda_2)$ shifts left with increasing $\alpha$.
\end{thm}
\begin{proof}
	The proof proceeds in analogy to \Cref{lem:node_left}. First, note that by \Cref{cor:ordering},
	\begin{equation}
		\left\langle \mathbf{U} \right\rangle_{\mathbf{u}(\lambda_2)} \geq m-1
	\end{equation}
	where $m$ corresponds to the generalized zero of $\mathbf{u}(\lambda_2)$. Then, like \cref{eq:node_theta}
	\begin{align}
		U_{m} - \left\langle \mathbf{U} \right\rangle_{\mathbf{u}(\lambda_2)} &= (m-1) - \left\langle \mathbf{U} \right\rangle_{\mathbf{u}(\lambda_2)} \leq 0.
	\end{align}
	Note that because $\mathbf{u}(\lambda_2)$ has at least one positive and one negative term, the inequality is strict when $m=1$. If $m>1$,
	\begin{equation}
		U_{m-1} - \left\langle \mathbf{\lambda_2} \right\rangle_{\mathbf{u}(\lambda_2)} < 0.
	\end{equation}
	Thus, by \cref{eq:theta_vary}
	\begin{equation}
	  \frac{d \Theta_{U,i}}{d\beta} <0 \; \; \; \text{$\forall i \in \intset{1}{m}$}.
	\end{equation}
	Hence, by the same logic as \Cref{lem:node_left}, \Cref{thm:Gantmakher} applies and the node always shifts left with increasing $\alpha$.
\end{proof}

